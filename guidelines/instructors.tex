
\section{Instructor Roles \& Responsibilities}
\label{sec:instructors}
\noindent
This bootcamp is a student-run program.
Unless instructors are desperately needed to keep the workload manageable, the
experience in curriculum design and technical instruction with little to no
oversight is best left for graduate students.
Among however many instructors there may be, one serves as the “chief” and is
therefore holistically responsible for overseeing the remaining instructors,
ensuring that the bootcamp happens, and maximizing the overall quality of the
instruction.
With a more or less fixed amount of work required to pull off one instance of
this bootcamp, it is absolutely in the regime of ``many helping hands make
light work.''

\subsection{The Chief}
\label{sec:instructors:chief}
\noindent
The chief’s role is largely delegatory; they are responsible for facilitating
every step outlined in section~\ref{sec:timeline} below.
For example, they divide up the work among the instructors (including
themselves) and supervise them to ensure the bootcamp goes smoothly.
The chief’s role is less participant-facing than other instructors, and
consequently their teaching skills are less important, but important
nonetheless.
The ideal chief is someone who satisfies each of the following criteria, in
order of importance:
\begin{enumerate}[topsep=0pt, itemsep=0pt, partopsep=0pt, parsep=0pt]

	\item They are sufficiently familiar with each of the topics covered (see
	section~\ref{sec:curriculum} above) that they would be able to teach any of
	them without help (\textit{essential}).

	\item They have experience as an instructor from one or more previous
	instances of this bootcamp under a prior chief (\textit{essential}).

	\item They are willing and able to attend each session in person
	(\textit{strongly recommended}).

	\item They are not enrolled in any coursework (\textit{strongly
	recommended}).

	\item They are not in the final year of their Ph.D. (\textit{recommended
	but not essential}).
\end{enumerate}
Criteria (1) and (2) are the most essential qualities of the chief, because
they serve as a backup instructor for all sessions.
If an instructor is suddenly unable to teach on a given day, the chief will
step in.
Furthermore, in the event that an instructor does not know the answer to a
participant’s question, the chief must track down the answer if they do not
know it off the top of their head.
It is also the chief’s duty to troubleshoot any procedural difficulties that
may arise, such as technical difficulties with room microphones or cameras for
participants joining virtually.
The unpredictable nature of exactly when the chief may be needed motivates
criterion (3).
\par
In practice, prior experience as an instructor (criterion 2) indicates that
most chiefs will have completed their second year of the astronomy Ph.D.
program and therefore will not be enrolled in any courses.
However, some graduate students take additional courses even after their second
year for a variety of reasons.
The ideal chief is not enrolled in any coursework (criterion 4) to minimize
competition for their time and mental energy during preparation and execution
of the bootcamp, which overlaps with the final ~6 weeks of the spring semester
(see section~\ref{sec:timeline} below).
Criterion (5), if satisfied, maximizes the continuity of knowledge between a
chief and their successor, though this recommendation is secondary to other
considerations.

\subsubsection{Succession}
\label{sec:instructors:chief:succession}
\noindent
The chief’s final duty following each instance of this bootcamp is to identify
their successor if they choose to relinquish the role.
Using their best judgment, they should identify which of the instructors best
meets the criteria outlined in section~\ref{sec:instructors:chief}.
They should consult with the SURP director (Dr. Wayne Schlingman at present) in
making this decision.
The current chief and the SURP director should then inform the person they have
chosen of their nomination.
If desired, the nominee may request a closed-door discussion with the SURP
director and the current chief about the role, what it entails, and most
importantly, any recommendations they have for the next instance of the
bootcamp that they would like to keep off the record.


