
\section{Instructor Roles \& Responsibilities}
\label{sec:instructors}
\noindent
This bootcamp is a student-run program.
With exposure to curriculum design and technical instruction with little
to no oversight in a low-stakes environment, the experiences offered are best
reserved for graduate students.
More senior scientists may join as instructors at the discretion of the chief
(see section~\ref{sec:instructors:chief} below) if they are needed to keep the
workload manageable.
With a more or less fixed amount of work required to pull off one instance of
this bootcamp, it is absolutely in the regime of ``many helping hands make
light work,'' so a collaborative approach is encouraged.

\subsection{The Chief}
\label{sec:instructors:chief}
\noindent
Among the instructors for one instance of this bootcamp, one serves as the
``chief.''
The chief is holistically responsible for the overall quality of the bootcamp
and is tasked with supervising their teammates.
If the audience of participants is the SURP class, then throughout the process,
the chief will work closely with the SURP director.
The SURP director can also serve as a valuable advisor to the chief even if the
participant audience is not a class of SURP students.
The chief (especially if they are a first-time chief) is also encouraged to use
their predecessors as resources, including those that may have left the
department.
The role is largely delegatory; the chief is responsible for facilitating
every step outlined in section~\ref{sec:timeline} below.
For example, they divide up the work among the instructors, which includes
themselves.
The chief’s role is less participant-facing than other instructors, and
consequently their teaching skills are less important, but important
nonetheless.
\par
The ideal chief is someone who satisfies each of the following criteria, in
order of importance:
\begin{enumerate}[topsep=0pt, itemsep=0pt, partopsep=0pt, parsep=0pt]

	\item They are sufficiently familiar with each of the topics covered (see
	section~\ref{sec:curriculum} above) that they could teach any of them
	without help (\textit{essential}).

	\item They have experience as an instructor from one or more previous
	instances of this bootcamp under a prior chief (\textit{essential}).

	\item They will not undergo any formal review or examination (including but
	not limited to their dissertation defense, thesis proposal, or Ph.D.
	candidacy exam) during preparation or execution of the bootcamp
	(\textit{essential}).

	\item They are not enrolled in any coursework (\textit{strongly
	recommended; essential if there are scheduling conflicts}).

	\item They are willing and able to attend each session in person
	(\textit{strongly recommended}).

	\item They are not in the final year of their Ph.D. (\textit{recommended
	but not essential}).
\end{enumerate}
Criteria (1) and (2) are the most essential qualities of the chief, because
they serve as a secondary/backup instructor for all sessions (see
section~\ref{sec:timeline:execution} below).
Criterion (3) is self-explanatory as these kinds of obligations inevitably
compete for one's time and mental energy, which the chief cannot afford.
Criterion (4) is similarly motivated but is less important since coursework is
more flexible than, e.g., the candidacy exam.
However, the level of concern should increase with the demands posed by the
courses in question since preparation overlaps with the final~$\sim$6 weeks of
the spring semester (see section~\ref{sec:timeline} below).
Criterion (5) is motivated by a combination of the unpredictable nature of when
exactly the chief may be needed (see section~\ref{sec:timeline:execution}), the
requirement that they oversee the team of instructors, and the delays in
communication that often arise when joining virtually.
Criterion (6), if satisfied, maximizes the continuity of knowledge between a
chief and their successor, though this recommendation is secondary to other
considerations.

\subsubsection{Succession}
\label{sec:instructors:chief:succession}
\noindent
The chief’s final duty following each instance of this bootcamp is to identify
their successor if they choose to relinquish the role.
Using their best judgment, they should identify which of the
instructors\footnote{
	Those who served as instructors in the past but were not involved in the
	most recent instance of this bootcamp~\textit{are} eligible for nomination.
} best meets the criteria outlined in section~\ref{sec:instructors:chief}.
They should consult with the SURP director in making this decision, regardless
of whether or not the previous audience of participants was the SURP class.
The current chief and the SURP director should then inform the chosen successor
of their nomination, which they can decline if they so choose.
If desired, the nominee may request a closed-door discussion with the SURP
director and the current chief about the role, what it entails, and most
importantly, any recommendations they may have that they would like to keep off
the record.

\subsection{Other Instructors}
\label{sec:instructors:others}
\noindent
Additional instructors volunteer for their positions in response to a call sent
by the chief sufficiently in advance of an instance of this bootcamp (see
section~\ref{sec:timeline:prep} below).
There are two general ways in which they can contribute to the team's efforts:
serving as a lecturer and constructing new and/or modifying the existing
curriculum.
Instructors can serve in one or both of these two roles however they and the
chief see fit.
The chief will assign lecture positions to the instructors based on their
judgment of who is best equipped for each topic.
The desire to teach a specific topic does not guarantee that the chief will
assign an instructor to that topic; there may be another instructor who is
better equipped, and the chief's ultimate priority is the quality of the
instruction provided for the participants.
\par
The role of lecturers is simple; they simply deliver the slides that the chief
has assigned them to during the corresponding session.
Though the chief is ideally someone who can attend in-person (see
section~\ref{sec:instructors:chief} above), the remaining instructors may join
virtually if need be.
However, in practice, this makes it difficult to engage with the students, so
lecturers are encouraged to attend in-person if at all possible (see also
section~\ref{sec:improvements} below).
The utmost priority of each lecturer is the clarity of their delivery.
It is imperative that they go into their assigned sessions knowing
\textit{exactly} what the example codes in their slides are doing on a
line-by-line basis.
Without a clear connection between the concepts at play and code in front of
them, the participants are unable to connect what they are hearing to the ways
it should influence how they write code.
As this detail of their delivery is perhaps the most central to the quality of
the instruction the participants receive, the chief's primary concern during
preparation will be that their lecturers understand the examples.
\par
Instructors who wish to revise or construct curriculum, a less
participant-facing role than lecturing, should discuss their ideas with the
chief, who reserves the right to the final say in whether it will be
implemented.
The chief also has the right to assign instructors to these tasks, which may
include revisions to curriculum, entirely new curriculum, or specific exercises
and solutions that they would like to provide for the participants.
In practice, those best equipped to teach entirely new material will be the
chief themselves and the person or people who constructed the slides.




% If an instructor is suddenly unable to teach on a given day, the chief will
% step in.
% Furthermore, in the event that an instructor does not know the answer to a
% participant’s question, the chief must track down the answer if they do not
% know it off the top of their head.
% It is also the chief’s duty to troubleshoot any procedural difficulties that
% may arise, such as technical difficulties with room microphones or cameras for
% participants joining virtually.
