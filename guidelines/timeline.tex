
\section{Timeline: Planning, Preparing \& Executing}
\label{sec:timeline}
\noindent
Preparation for a given instance of a bootcamp should begin about six weeks
ahead of the nominal start date.
The chief should email the astronomy graduate students with a call for
instructors at least twice (once for the initial call, and again for everyone
who forgets to respond).
The chief should have a short conversation with each instructor about their
skill level to assess what they are most capable of teaching, and viewing code
they have written may be a necessary component of this meeting.
It is good practice for the chief to remind them that this conversation is not
a test or exam of any sort, but that they simply need to make an informed
decision about who is best equipped to teach what topics.
During this time, the chief should also assess what changes need made to the
curriculum.
\par
If the audience of participants is the SURP class, then this call for
instructors should be sent out in the final week of March.
This allows sufficient time for preparation during the month of April before
SURP begins in the first full week of May.
Although this unfortunately overlaps with the end of spring semester classes
for the first- and second-year graduate students, this reality is unavoidable
as the bootcamp must be delivered at the beginning of SURP and it simply takes
time to prepare.
Because much of the curriculum is already in place, many of the instructors do
not need to construct new material, though this may be desirable or small
modifications may be required, e.g., because there was a new version of python
since the last instance of the bootcamp.
In general, the task of constructing wholly new material and/or exercises is
best reserved for instructors~\textit{not} enrolled in classes as this demand
on an instructor's time may make it difficult to have the material in place
when the bootcamp begins.

\subsection{Preparation}
\label{sec:timeline:prep}
\noindent
The chief should check in with their team of instructors periodically
throughout the weeks leading up to the bootcamp itself.
These check-ins should be more frequent with instructors who are designing
curriculum, especially if it is entirely new as opposed to a modification from
previous years.
In practice, a good rule of thumb is to ensure that all lecturers have
sufficiently familiarized themselves with their assigned slides two weeks
before the start of the bootcamp.
By ensuring familiarity early, the chief can fill any gaps in the lecturers'
knowledge as needed and avoid a time crunch.
The chief should pay particular attention to how well each lecturer understands
the example codes on~\textit{every} slide as this detail is extremely central
to the quality of their instruction (see section~\ref{sec:instructors:others}
above).
A similar time frame of two weeks is recommended for those designing
curriculum, whether it is in the form of slides or exercises.
All material should be in place far enough in advance of the bootcamp itself so
that the chief can recommend any revisions they see fit.

\subsection{Execution}
\label{sec:timeline:execution}
\noindent
On the day of the first session, the chief should also advertise the bootcamp
at Astronomy Coffee.
For each session, including the first, the chief should send an announcement
with the time, place, zoom information, and a brief ($\sim$1 - 2
sentences) summary of the material covered to the astronomy and physics
departments shortly before lunch.
The chief should then help set up the computer that the lecturers will present
from and ensure that they can see their presenter notes shortly before they
begin.
They should also ensure that the zoom setup is working properly - that those
joining virtually can clearly hear them speaking and can see the slides - and
then begin recording the session.
\par
Once the participants and lecturers are ready, the chief should remind the
participants what was covered in the previous session (in the first session,
the chief and each lecturer simply introduce themselves; see
section~\ref{sec:curriculum:one} above).
They do not need to summarize the material, but simply state what it entailed,
and then ask for any questions on those topics before continuing with the day's
agenda.
Each session should have a built-in~$\sim$5-minute break as close to halfway
through as possible while also occurring at a natural stopping point between
different topics.
Following the instruction itself, the chief should make the participants aware
of where they can find the exercises and solution sets for that session.
Lecturers should remain in the classroom to answer questions and help the
participants who may be working on the exercises.
Once the recording of the session is available, either the chief or the SURP
director should email a copy of the recording to both the astronomy and physics
departments.
\par
Roughly one week following each bootcamp, the chief and the SURP director
should have a closed-door discussion with the instructors to debrief.
This meeting provides the instructors with a formal channel through which they
can provide feedback on how they think the bootcamp went overall and suggest
any ideas for improvements that they may have (see also
section~\ref{sec:improvements} below).

