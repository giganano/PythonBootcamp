
\section{Future Improvements}
\label{sec:improvements}
\noindent
With each passing instance of this bootcamp, we have new ideas about how we
might make it better.
This section simply provides a point-by-point overview of these ideas, in
no particular order.
Future chiefs should use this list for inspiration and guidance on how they
might adjust the program to better serve its participants.
\begin{itemize}

	\item Creating a Carmen page for the bootcamp.
	The SURP students are quite familiar and comfortable with Carmen, so
	receiving copies of the slides and exercises there would be seamless for
	them.

	\item A classroom more conducive to the program.
	Historically, the bootcamp has been held in McPherson 1010.
	Although this room is good for the SURP students to work in, it separates
	them from the instructors with ``levels'' and stairs, making it difficult
	to engage.
	The zoom setup in this room also leaves something to be desired.

	\item To survey the participants at the end of each bootcamp and again at
	the end of the summer.
	This would allow the team to more quantitatively judge what went well and
	where there is the most room for improvement.

	\item Quick exercises (akin to ``clicker questions'') throughout the
	slides.
	A similar idea is to fill the remaining time by doing exercises as a class.
	In the May 2020, May 2021, and May 2022 instances of this bootcamp, we paid
	little attention to the students activities on their computers during each
	lecture.
	In the May 2023 instance, we asked that they not try to follow along with
	the example codes for a variety of reasons that arise in practice.
	However, it is possible that this decision led to a less engaged audience.

	\item Debugging strategies as a topic of instruction.
	This goal was slated to be implemented in the May 2023 instance of this
	bootcamp, but the goal was not met in time due to a handful of factors.
	This idea would pair well with the ``clicker questions'' idea above.

	\item To follow the solution to a generic programming problem through the
	sessions (i.e., a solution that ``evolves'' with each new technique
	introduced).
	The problem, by nature, must naturally owe itself to~\textit{both}
	functional~\textit{and} object-oriented programming solutions.
	The deck of cards has been mentioned as one potential option, though it
	does not lend itself well to a functional programming solution, and a
	``final project'' type problem of some sort is an integral component of
	the exercises.

	\item The dates and times of each session should be pinned down early
	(specifically, even before the call for instructors goes out~$\sim$6 weeks
	before the bootcamp starts).
	This allows everyone potentially interested in joining the team to better
	judge their availability.

\end{itemize}
