
\section{Curriculum Overview}
\label{sec:curriculum}
\noindent
Each instance of this bootcamp includes six sessions that typically last
between 90 minutes and 2 hours.
At the beginning of each session, the chief (see
section~\ref{sec:instructors:chief}) should remind the participants what was
covered in the previous session (they do not need to summarize the material
itself, but simply state what it entailed) and ask for questions on that
material before continuing with the day’s agenda.
Each session should have a built-in ~5-minute break as close to halfway through
as possible while also happening at a natural stopping point between different
topics.
Following instruction itself, the instructors should remain in the classroom to
answer questions and help students who may be working on the exercises.
\par
The course material itself is hosted online in a GitHub repository.\footnote{
	\url{https://github.com/giganano/PythonBootcamp.git}
}
Each set of slides, exercises, and example codes can be found under directories
at the top level with the corresponding self-explanatory names.

\subsection{Session One}
\label{sec:curriculum:one}
\noindent
\textit{Slides: intro.pptx, context.pptx, terminal.pptx (in that order)}
\par\noindent
The first session begins with all instructors introducing themselves;
attendance is therefore mandatory for all instructors who are available, though
those who are not teaching that day may leave after introductions if they so
choose.
\par
This session begins by familiarizing participants with high-level details of
python as a programming language.
It draws comparisons between python and C to describe interpreted versus
compiled languages and between python and JavaScript to describe strongly
versus weakly typed languages.
It then switches topics to familiarize participants with the Unix terminal,
including aliases and environment variables.
After slides, instructors help the participants running Windows computers set
up the Windows Subsystem for Linux.

\subsection{Session Two}
\label{sec:curriculum:two}
\noindent
\textit{Slides: review.pptx, fileIO.pptx (in that order)}
\par\noindent
The second session is a crash-course review of functional programming in
python, including the material the participants are generally familiar with
combined with some nuance thereof.
This review includes the built-in data types (e.g., int, float, list, tuple),
loops, conditional statements, and functions.
It includes the common mistake made by junior programmers that lists and arrays
are not the same thing.
It then covers exception handling, which is less commonly known by the
participants at the beginning.
In the May 2022 instance of this bootcamp, we added coverage of the new
match-case syntax introduced in python 3.10, which implements pattern-matching
features.
This session concludes by describing how to read and write files without NumPy.

\subsection{Session Three}
\label{sec:curriculum:three}
\noindent
\textit{Slides: imports.pptx, rst.ppx, anaconda.pptx, anaconda\_usecases.pptx
(in that order)}
\par\noindent
This session concludes functional programming by covering multi-file programs
and popular third-party libraries available through Anaconda.
We begin by covering import statements and what python itself does behind the
scenes when code is imported.
We then use this as a base for describing how to set up the basic structure of
a python package.
We then cover a handful of popular use cases of Anaconda within astronomy,
material which Joy Bhattacharyya expanded in the May 2023 instance of this
bootcamp.

\subsection{Session Four}
\label{sec:curriculum:four}
\noindent
\textit{Slides: classes.pptx}
\par\noindent

\subsection{Session Five}
\label{sec:curriculum:five}
\noindent
\textit{Slides: inheritance\_composition.pptx}
\par\noindent

\subsection{Session Six}
\label{sec:curriculum:six}
\noindent
\textit{Slides: basicSE.pptx, github.pptx (in that order)}
\par\noindent

