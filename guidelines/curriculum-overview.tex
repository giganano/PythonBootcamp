
\section{Curriculum Overview}
\label{sec:curriculum}
\noindent
Each instance of this bootcamp includes six sessions that typically last
between 90 minutes and 2 hours.
This section provides a brief overview of the topics covered broken down by
session.
The course material itself is hosted online in a GitHub repository.\footnote{
	\url{https://github.com/giganano/PythonBootcamp.git}
}
Each set of slides, exercises, and example codes can be found under directories
at the top level with the corresponding self-explanatory names.
As a public repository, it can be cloned or viewed online by anyone.
However, making changes to the source material (e.g., to add or make
modifications to existing slides or exercises), requires either the current
chief (see section~\ref{sec:instructors:chief}) or the SURP director (Dr. Wayne
Schlingman) to add one as a collaborator to the repository.
Though collaborative tools such as Google Slides may seem like attractive
alternative options for hosting the material, usage of GitHub ensures that a
significant subset of the instructors are familiar with the material covered in
session six (see section~\ref{sec:curriculum:six} below).
Slides, exercises, and recordings of each session from previous years can also
be found at~\url{https://jamesjohnson.space/bootcamp}.

\subsection{Session One}
\label{sec:curriculum:one}
\noindent
\textit{Slides: intro.pptx, context.pptx, terminal.pptx (in that order)}
\par\noindent
The first session begins with all instructors introducing themselves;
attendance is therefore mandatory for all instructors who are available, though
those who are not teaching that day may leave after introductions if they so
choose.
\par
This session then familiarizes participants with high-level details of
python as a programming language.
It draws comparisons between python and C to describe interpreted versus
compiled languages and between python and JavaScript to describe strongly typed
versus weakly typed languages.
It then switches topics to familiarize participants with the Unix terminal,
including aliases and environment variables.
After the slides are finished, instructors are to help the participants running
Windows computers with setting up the Windows Subsystem for Linux.
By taking care of this final step in session one, each of the students is then
running code in a Unix environment, which both minimizes difficulty for the
instructors and eases participants' transitions to supercomputing facilities if
their research demands it.

\subsection{Session Two}
\label{sec:curriculum:two}
\noindent
\textit{Slides: review.pptx, fileIO.pptx (in that order)}
\par\noindent
The second session is a crash-course review of functional programming in
python, including the material the participants are generally familiar with,
but with some additional nuance thereof.
This review includes the built-in data types (e.g.,~\textit{int},
\textit{float},~\textit{list},~\textit{tuple}), loops, conditional statements,
and functions.
It includes the common mistake made by junior programmers that~\textit{list}s
and NumPy~\textit{array}s are not the same thing.
It then covers exception handling, which is less commonly known by the
participants at the beginning.
In the May 2022 instance of this bootcamp, we added coverage of the new
match-case syntax introduced in python 3.10, which implements pattern-matching
algorithms.
This session concludes by describing how to read and write files without NumPy.

\subsection{Session Three}
\label{sec:curriculum:three}
\noindent
\textit{Slides: imports.pptx, rst.ppx, anaconda.pptx, anaconda\_usecases.pptx
(in that order)}
\par\noindent
This session concludes functional programming by covering multi-file programs
and popular third-party libraries available through Anaconda.
We begin by covering~\textit{import} statements and what python itself does
behind the scenes when code is~\textit{import}ed.
We then use this as a base for describing how to set up the basic structure of
a python package (i.e., the~\textit{\_\_init\_\_.py} file).
We then cover a handful of popular use cases of Anaconda within astronomy,
material which Joy Bhattacharyya substantially expanded in the May 2023
instance of this bootcamp.
This session is accompanied by exercises intended to familiarize the
participants with some essential tasks as researchers, such as making figures
with matplotlib and parsing documentation.

\subsection{Session Four}
\label{sec:curriculum:four}
\noindent
\textit{Slides: classes.pptx}
\par\noindent
This session covers the first half of the object-oriented programming
instruction.
We begin by introducing simple data container objects, after which we describe
how to implement objects that take parameters upon initialization as well as
property and setter functions to implement error handling of object attributes.
After briefly discussing~\textit{classmethod}s and~\textit{staticmethod}s, we
then spend the rest of this session describing syntactic sugar (sometimes
referred to as ``magic methods'' in this context) and emulating numeric types
(i.e., making the object respond to the, e.g.,~\textit{+} and~\textit{-}
operators).
We walk the participants through code implementing generic polynomial functions
as an example of one such object, and their primary exercise from this material
is to implement a generic vector object.

\subsection{Session Five}
\label{sec:curriculum:five}
\noindent
\textit{Slides: inheritance\_composition.pptx}
\par\noindent
This session covers inheritance and composition, the second half of the
object-oriented programming instruction.
The majority of the time is spent discussing inheritance as it is the one of
these two techniques which introduces new syntax.
We discuss subclassing both user-implemented and built-in types before
describing how the python data model presents itself when using inheritance
(i.e., everything inherits from~\textit{object}).
We conclude this session with composition, which introduces no new syntax and
is therefore much simpler than inheritance.
We create an object storing data on solar system bodies in an intuitive
structure to describe the utility of composition.

\subsection{Session Six}
\label{sec:curriculum:six}
\noindent
\textit{Slides: basicSE.pptx, github.pptx (in that order)}
\par\noindent
The final session begins by covering some basic principles of software
engineering that reinforce good coding habits.
In practice, the SURP students do not realize the usefulness of these concepts
until later in the summer (mid/late July) when the code bases of their research
projects become sufficiently large.
This session then concludes with an introduction to GitHub, applications
thereof that are useful to scientists, and the terminal
application~\textit{git}.

\subsection{Exercises}
\label{sec:curriculum:exercises}
\noindent
Each session, except for the sixth, is accompanied by a set of exercises for
the participants.
Their work is never collected by the instructors as this bootcamp is not a
graded course.
Constructing a solution is to the benefit of each participant and can be done
on their own time, which may not be until later in the summer when they feel
they have a sufficient grasp of the material to try their hand at the problems.
These exercises are available alongside the slides and example codes in the
same GitHub repository.
They are sorted into sub-directories by session, with correct solutions
provided therein, paired as follows:
\begin{itemize}

	\item \textbf{Session 1}:~\textit{exercises/terminal}

	\item \textbf{Session 2}:~\textit{exercises/review}

	\item \textbf{Session 3}:~\textit{exercises/imports} and
	\textit{exercises/documentation}

	\item \textbf{Session 4}:~\textit{exercises/classes}

	\item \textbf{Session 5}:~\textit{exercises/inheritance\_composition}

\end{itemize}
Links to individual exercises and solutions broken down by session can also be
found at \\ \url{https://jamesjohnson.space/bootcamp}.
One exercise that the chief (see section~\ref{sec:instructors:chief}) should
draw the participants' attention to at the end of session five is the deck
of cards, located at~\textit{exercises/inheritance\_composition/cards.txt}.
Intended to be a ``final project'' of sorts, this programming problem is too
extensive to be solved in one sitting.
It requires the participants to think about their algorithms and to have a
``plan of attack'' with an organized workflow while simultaneously bringing in
concepts from each of the sessions leading up to this point.
In general, successful completion of the deck of cards is an indication that
one has a solid grasp on all of the concepts covered by this bootcamp.
Every participant is encouraged to try their hand at it once they feel they are
ready.

