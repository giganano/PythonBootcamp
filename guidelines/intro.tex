
\section{Introduction}
\label{sec:intro}
\noindent
This bootcamp is aimed at scientific researchers who are familiar with
functional programming in python but are not necessarily experts therein.
Functional programming in general encompasses all practices and techniques
short of object-oriented programming, i.e., functions (\textit{def}), loops
(\textit{for} and~\textit{while}), and conditional statements (\textit{if},
\textit{elif}, and~\textit{else}).
Though exception handling (\textit{try},~\textit{except}, and~\textit{finally})
and multi-file programs technically fall under this category as well,
researchers are less familiar with these techniques in practice.
They are therefore covered to the appropriate level of detail along with a
review of the rest of functional programming.
\par
The ultimate goals of this curriculum are to review the concepts of functional
programming and introduce the students to the Unix terminal, multi-file
programs, python packages, and object-oriented programming.
As a bootcamp, the material is much more condensed than if it were presented in
the format of a traditional semester-long course.
Consequently, the instruction proceeds sufficiently fast that most students are
still learning a given set of concepts and techniques before the next session
presents them with an entirely new set.
Mastery is therefore almost never achieved in the timespan of the bootcamp
itself and instead comes over the course of the summer as the students apply
the concepts they learn to the code bases for their research projects.
The best practice is to make this reality clear to the participants early and
remind them of it often.
\par
In practice, the target audience for this bootcamp is most often the
participants of the Summer Undergraduate Research Program (SURP).
Each year, SURP is dominated by students entering their third and fourth years
of the major.
Most of these students have received some level of instruction in python from
ASTRON 1221 (\textit{Astronomy Data Analysis}) and ASTRON 3350 (\textit{Methods
of Astronomical Observation \& Data Analysis}), which places the typical
student at exactly the level of familiarity with functional programming
appropriate for this bootcamp.
With instruction in object-oriented programming, however, the curriculum has
proved useful for scientists of all levels of seniority, including tenured
faculty members (whether or not they have the time to participate is, however,
a different question entirely).
As a result, each session run for SURP is open to participation by all members
of the Departments of Physics and Astronomy.
The instructional material can be easily modified to accommodate an audience
dominated by more experienced researchers (e.g., graduate students) by simply
shifting emphasis between different topics.

